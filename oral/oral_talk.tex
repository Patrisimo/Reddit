% !TEX TS-program = pdflatex
% !TEX encoding = UTF-8 Unicode

% This file is a template using the "beamer" package to create slides for a talk or presentation
% - Giving a talk on some subject.
% - The talk is between 15min and 45min long.
% - Style is ornate.

% MODIFIED by Jonathan Kew, 2008-07-06
% The header comments and encoding in this file were modified for inclusion with TeXworks.
% The content is otherwise unchanged from the original distributed with the beamer package.

\documentclass{beamer}


% Copyright 2004 by Till Tantau <tantau@users.sourceforge.net>.
%
% In principle, this file can be redistributed and/or modified under
% the terms of the GNU Public License, version 2.
%
% However, this file is supposed to be a template to be modified
% for your own needs. For this reason, if you use this file as a
% template and not specifically distribute it as part of a another
% package/program, I grant the extra permission to freely copy and
% modify this file as you see fit and even to delete this copyright
% notice. 


\mode<presentation>
{
  \usetheme{Warsaw}
  % or ...

  \setbeamercovered{transparent}
  % or whatever (possibly just delete it)
}


\usepackage[english]{babel}
% or whatever

\usepackage[utf8]{inputenc}
% or whatever

\usepackage{times}
\usepackage[T1]{fontenc}
% Or whatever. Note that the encoding and the font should match. If T1
% does not look nice, try deleting the line with the fontenc.
\usepackage{mathtools}
\usepackage{amsmath}
\usepackage{dsfont}

\title % (optional, use only with long paper titles)
{Learning Theory for Classification}

\author % (optional, use only with lots of authors)
{Patrick Martin}
% - Use the \inst{?} command only if the authors have different
%   affiliation.

\institute % (optional, but mostly needed)
{Department of Mathematics\\
Johns Hopkins University}
% - Use the \inst command only if there are several affiliations.
% - Keep it simple, no one is interested in your street address.

\date % (optional)
{Oral Exam - April 3, 2018}

\subject{Oral Exam}
% This is only inserted into the PDF information catalog. Can be left
% out. 



% If you have a file called "university-logo-filename.xxx", where xxx
% is a graphic format that can be processed by latex or pdflatex,
% resp., then you can add a logo as follows:

% \pgfdeclareimage[height=0.5cm]{university-logo}{university-logo-filename}
% \logo{\pgfuseimage{university-logo}}



% Delete this, if you do not want the table of contents to pop up at
% the beginning of each subsection:

\AtBeginSection[]
{
  \begin{frame}<beamer>{Outline}
    \tableofcontents[currentsection]
  \end{frame}
}



% If you wish to uncover everything in a step-wise fashion, uncomment
% the following command: 

%\beamerdefaultoverlayspecification{<+->}

\renewcommand{\Pr}[1]{\mathcal{P} \left( #1 \right)}
\newcommand{\cls}{\mathcal{C}}
\newcommand{\E}[1]{\mathbb{E}\left( #1 \right)}
\newcommand{\R}{\mathbb{R}}
\newcommand{\defeq}{\vcentcolon=}
\newcommand{\eqdef}{=\vcentcolon}
\newcommand{\pa}[1]{\left( #1 \right)}
\newcommand{\I}{\mathds{1}}
\newcommand{\abs}[1]{\left| #1 \right|}
\newcommand{\br}[1]{\left\{ #1 \right\} }
\newcommand{\eps}{\varepsilon}
\newcommand{\an}[1]{\langle #1 \rangle}

\begin{document}

\begin{frame}
  \titlepage
\end{frame}

\begin{frame}{Outline}
  \tableofcontents
  % You might wish to add the option [pausesections]
\end{frame}


% Since this a solution template for a generic talk, very little can
% be said about how it should be structured. However, the talk length
% of between 15min and 45min and the theme suggest that you stick to
% the following rules:  

% - Exactly two or three sections (other than the summary).
% - At *most* three subsections per section.
% - Talk about 30s to 2min per frame. So there should be between about
%   15 and 30 frames, all told.

\section{Introduction}
\begin{frame}{What is Classification?}
\begin{itemize}
\item Given some \textcolor{blue}{symptoms}, what \textcolor{red}{illness} does a patient have?
\item Given an \textcolor{blue}{English document}, what \textcolor{red}{dialect} is it written in? 
\item Given an \textcolor{blue}{email}, is it \textcolor{red}{spam}?
\item Given some \textcolor{blue}{observables}, what \textcolor{red}{class} produced it?
\end{itemize}
\end{frame}

\begin{frame}{Giving some names}
\begin{itemize}
\item Our \textcolor{blue}{observables} will be feature vectors $X \in \R^d$
\item Our \textcolor{red}{classes} will be numbers $Y=0,\ldots,k-1$
\item Thus our data is ordered pairs $(X,Y) \in \R^d \times \{0,\ldots,k-1\}$
\item A classifier is then a \emph{function}
\[ g: \R^d \to \{0,\ldots, k-1\} \]
\end{itemize}
\end{frame}

\begin{frame}{Data Generation - Part 1}
\begin{itemize}
\item Our first paradigm of how the data is generated might be the following
\begin{block}{Mixture model}
Each class $Y$ represents a probability distribution over $\R^d$, and so data is generated by first choosing a class, and then generating a vector $X$ based on that class's distribution $\Pr{X|Y}$.\\
\end{block}
\item Given a vector $X$, the probability that it belongs to a certain class is then given by Bayes' Rule:
\[ \Pr{Y|X} \propto \Pr{X|Y} \Pr{Y} \]
\end{itemize}
\end{frame}

\begin{frame}{Data Generation - Part 2}
\begin{itemize}
\item However, we really only care about the $\Pr{Y|X}$, and so we can instead describe our data differently
\begin{block}{Classifier model}
Data is generated by choosing vectors $X$, and then assigning a class $Y$ according to $\Pr{Y|X}$. 

\end{block}
\end{itemize}
\end{frame}

\begin{frame}{Evaluation}
\begin{itemize}
\item A classifier $g$ makes an \emph{error} for $(X,Y)$ if $g(X) \ne Y$
\item We can then talk about the \emph{probability of error} $L$, in terms of the probability distribution $\nu$:
\[ L(g) \defeq \Pr{g(X) \ne Y} = \nu\pa{ \br{ (X,Y) : \phi(X) \ne Y}}  \]
\item The best classifier is the one that minimizes the probability of error
\[ g^* = \arg \min_g L(g) \]
\item This $g^*$ is called the \emph{Bayes classifier}, and $L^* \defeq L(g^*)$ is the \emph{Bayes error}
\end{itemize}
\end{frame}

\begin{frame}{Ready to go?}
\begin{itemize}
\item Unfortunately we can't really find $g^*$ exactly, because
\begin{itemize}
\item The space of functions $g: \R^d \to \{0,\ldots,k-1\}$ is too large and unstructured to search
\item We generally don't know $\Pr{X,Y}$, and can't actually compute $L(g)$
\end{itemize}
\end{itemize}
\end{frame}

\begin{frame}{Simplifications}
\begin{itemize}
\item Instead, we will restrict our search to a subset of classifiers $\cls$
\item We will use training data to estimate $L$ and $\nu$
\item For simplicity, we will also restrict ourselves to binary classification ($k=2$)
\end{itemize}
\end{frame}

\begin{frame}{Questions}
\begin{itemize}
\item How well does the best classifier in our class do?
\item How well can we approximate the best classifier?
\item How much training data do we need to do this approximation?
\end{itemize}
\end{frame}

\begin{frame}{Estimations and Approximations}
\begin{itemize}
\item Introduce the \emph{empirical} error probability for a classifier
\[ \hat{L}_n(g) \defeq \frac1n \sum_{i=1}^n \mathds{1}_{g(X_i) \ne Y_i} = \nu_n\pa{\br{(X,Y) : \phi(X) \ne Y}}\]
\item If we restrict our search to $\phi \in \cls$, then we can choose $\phi_n^*$ to minimize $\hat{L}_n$
\item How close to the Bayes error can we get?
\[ L(\phi_n^*) - L^* = \underbrace{\pa{ L(\phi_n^*) - \inf_{\phi \in \cls} L(\phi)}}_{\text{estimation error}} + \underbrace{\pa{ \inf_{\phi \in \cls} L(\phi) - L^*}}_{\text{approximation error}} \]
\item We can try to control the estimation error, but the approximation error belongs to $\cls$
\end{itemize}
\end{frame}

\section{Bounding the Estimation Error}
\subsection{Finite $\cls$}
\begin{frame}{Finite $\cls$}
\begin{block}{Finite $\cls$ - Lucky guess}
If $\abs{\cls} < \infty$ and $\min_{\phi \in \cls} L(\phi) = 0$, then for every $n$ and $\eps > 0$,
\[ \Pr{L(\phi^*_n) > \eps} \leq \abs{\cls} e^{-n\eps}\]
and
\[ \E{L(\phi^*_n)} \leq \frac{1 + \log\abs{\cls}}n \]
\end{block}
\begin{itemize}
\item However, this only holds if a ``perfect'' classifier is in our $\cls$
\end{itemize}
\end{frame}

\begin{frame}{Proof of Probability Bound- Sketch}
\begin{align*}
\Pr{ L(\phi^*_n) > \eps } & \leq \Pr{ \max_{\phi \in \cls, \hat{L}_n(\phi) = 0} L(\phi) > \eps} \\
	& = \E{ \mathds{1}_{\max_{\br{\phi \in \cls, \hat{L}_n(\phi) = 0}} L(\phi) > \eps}} \\
	& = \E { \max_{\phi \in \cls}\pa{ \mathds{1}_{ L(\phi) > \eps} \mathds{1}_{\hat{L}_n(\phi) = 0}}} \\
	& \leq  \sum_{\phi \in \cls, L(\phi) > \eps} \Pr{ \hat{L}_n (\phi) = 0}  \\
	& \leq \abs{\cls} \pa{1-\eps}^n \\
	& \leq \abs{\cls} e^{-n\eps} \end{align*}
\end{frame}

\begin{frame}{Proof of Expectation - Sketch}
\begin{align*}
\E{L(\phi^*_n)} &= \int_0^\infty \Pr{L(\phi^*_n) > t} dt \\
	&= \int_0^u \Pr{L(\phi^*_n) > t} dt + \int_u^\infty \Pr{L(\phi^*_n) > t} dt \\
	& \leq u + \int_u^\infty \abs{\cls}e^{-nt} dt \\
	& = u + \frac{\abs{\cls}}n e^{-nu}\\
	& = \frac{ nu+ \abs{\cls}e^{-nu}}n
\end{align*}
The right hand side is minimized for $u = \frac{\log\abs{\cls}}n$, giving the desired bound
\end{frame}


\begin{frame}{Finite $\cls$ - General case}
\begin{itemize}
\item Turning our attention from $L$ to $\nu$, a classifier represents a subset of $\R^d \times \{0,1\}$ 
\[ \phi \to A \defeq \br{ (X,Y) : \phi(X) \ne Y}\]
\begin{block}{Finite $\cls$ - Unlucky guess}
If a class of sets $\mathcal{A}$ has finite cardinality, 
\[ \Pr{ \sup_{A\in \mathcal{A}} \abs{\nu_n(A) - \nu(A)} > \eps} \leq 2\abs{\mathcal{A}} e^{-2n\eps^2} \]
\end{block}
\item Notice that having a ``perfect'' classifier in $\cls$ improves our bound!
\end{itemize}
\end{frame}

\subsection{Fingering}
\begin{frame}{Effectively finite $\cls$}
\begin{itemize}
\item While $\cls$ will usually be infinite, it may be that there is a number $k$ (the \emph{fingering dimension}) and a function $\Psi: (\R^d)^k \to \cls$ such that for any $x_1,\ldots,x_n$ the behavior of any $\phi \in \cls$ can be replicated almost surely, with at most $k$ mistakes, that is
\[ \Psi(x_{i_1}, \ldots, x_{i_k})(x_j) = \phi(x_j) \quad \forall j \not\in \{i_1,\ldots,i_k\} \]
\item Fingering dimension of various types of classifiers
\begin{itemize}
\item Linear classifiers: $d$
\item Hyperrectangular classifiers: $2d$
\item Spherical classifiers: $d+1$
\end{itemize}
\item In this case, for any training data we only have to look at finitely many classifiers, at most $\frac{n!}{(n-k)!}$ many.
\end{itemize}
\end{frame}

\begin{frame}{Bounds}
\begin{block}{Bound on classifier selected by fingering}
If $\cls$ has fingering dimension $k$ and $\hat\phi$ is found by fingering, then for $n \geq k$ and $\frac{2k}n \leq \eps \leq 1$
\[ \Pr{ L(\hat\phi) - \inf_{\phi \in \cls} L(\phi) > \eps} \leq (n^k + 1)e^{-n\eps^2/2} e^{2k\eps} \]
and
\[ \E{ L(\hat\phi) - \inf_{\phi \in \cls} L(\phi)} \leq \sqrt{\frac{(2k+1)\log n + (2k+2)}n}\]
\end{block}
\end{frame}

\begin{frame}{Proof of Bounds - Sketch}
\begin{align*}
\Pr{ L(\hat\phi) - \inf_{\phi\in\cls} L(\phi)} & \leq \sum_{i=1}^{n!/(n-k)!} \Pr{L(\phi_i) - \hat{L}_n(\phi_i) > \eps/2} \\
	& \quad + \Pr{ \sup_{\phi\in \cls} \pa{ \hat{L}_n(\phi) - L(\phi) } > \eps/2 - k/n } \\
	& \leq \sum_{i=1}^{n!/(n-k)!} e^{-n\eps^2/2} e^{2k\eps} + e^{-n\eps^2/2} e^{2k\eps}\\
	& \leq (n^k +1)e^{-n\eps^2/2} e^{2k\eps}
\end{align*}

\end{frame}

\subsection{VC Dimension}
\begin{frame}{Shatter Coefficient}
\begin{itemize}
\item Another way of assigning a number to a class $\cls$ is by looking at the shatter coefficient and the VC dimension
% \item Recall that a class of classifiers $\cls$ induces a class of sets $\mathcal A$
\begin{block}{Shatter Coefficient}
The $n$-th \emph{shatter coefficient} of a class $\cls$ is
\[ S(\cls, n) \defeq \max_{(z_1,\ldots,z_n)\in (\R^d)^n} \abs{ \br{ \br{z_1, \ldots, z_n} \cap \phi^{-1}\pa{\br{1}} : \phi \in \cls}} \]
\end{block}
\item Immediately we see that $s(\cls, n) \leq 2^n$
\end{itemize}
\end{frame}


\begin{frame}{Bounds}
\begin{block}{Bound using Shatter Coefficient}
\[ \Pr{ \sup_{\phi \in \cls} \abs{ \hat{L}_n (\phi) - L(\phi)} > \eps } \leq 8 S(\cls, n) e^{-n\eps^2/32} \]
Letting $\phi_n^*$ be a classifier minimizing $\hat{L}_n(\phi)$ over $\cls$,
\[ \Pr{ L(\phi^*_n) - \inf_{\phi \in \cls} (\phi) > \eps } \leq 8 S(\cls,n)e^{-n\eps^2/128} \]
Also
\[ \E{L(\phi^*_n)} - \inf_{\phi\in\cls} L(\phi) \leq 16 \sqrt{\frac{\log(8eS(\cls, n))}{2n}} \]
\end{block}
\begin{itemize}
\item Notice that this is only helpful if $S(\cls,n) \ll 2^n$
\end{itemize}
\end{frame}

\begin{frame}{Proof of Bounds - Sketch}
\begin{enumerate}
\item Use a ``ghost sample'' $Z_1', \ldots, Z_n'$ to see that
\[ \Pr{ \sup_{A \in \mathcal{A}} \abs{\nu_n(A) - \nu(A)} > \eps} \leq 2\Pr{ \sup_{A \in \mathcal{A}} \abs{\nu_n(A) - \nu_n'(A)} > \eps/2}\]
\item Throw in random signs $\sigma_1,\ldots,\sigma_n$,
\[ \Pr{ \abs{\nu_n(A^*) - \nu(A^*)} > \eps} \leq 4 \Pr{\sup_{A\in\mathcal{A}} \frac1n\abs{\sum_{i=1}^n \sigma_i \mathds{1}_{Z_i \in A}} > \eps/4 }\]
\item Bound with Shatter coefficient
\begin{align*}
\Pr{\left.\sup_{A\in\mathcal{A}} \frac1n\abs{\sum_{i=1}^n \sigma_i \mathds{1}_{Z_i \in A}} > \eps/4 \right| Z_1,\ldots, Z_n} \quad\quad\quad\quad\quad\quad\quad\\
\quad \leq S(\mathcal{A}, n)\sup_{A \in \mathcal{A}} \Pr{\left.\frac1n \abs{\sum_{i=1}^n \sigma_i \mathds{1}_{Z_i \in A}} > \eps/4 \right| Z_1,\ldots,Z_n} \end{align*}
\end{enumerate}
\end{frame}

\begin{frame}{Proof of Bounds - Sketch}
\begin{enumerate}
\setcounter{enumi}{3}
\item Hoeffding and expectation
\[ \E{\Pr{\left.\frac1n \abs{\sum_{i=1}^n \sigma_i \mathds{1}_{Z_i \in A}} > \eps/4 \right| Z_1,\ldots,Z_n}} \leq 2e^{-n\eps^2/32} \]
\item Union Bound
\begin{align*}
  \Pr{ L(\phi^*_n) - \inf_{\phi \in \cls}L(\phi) > \eps} & \leq 2\Pr{ \sup_{A \in \mathcal{A}} \abs{\nu_n(A) - \nu(A)} > \eps/2} \\
	& \leq 8 S(\cls,n)e^{-n\eps^2/128}\end{align*}
\end{enumerate}
\end{frame}

\begin{frame}{VC Dimension}
\begin{itemize}
\item Let $k \geq 1$ be the largest integer such that
\[ s(\mathcal A, k) = 2^k \]
This is the \emph{VC dimension} of $\mathcal A$ (or $\cls$), which will also be denoted $V_{\mathcal A}$ (or $V_\cls$)
\begin{block}{Bound on Shatter Coefficient}
If $V_\cls > 2$, then $S(\cls, n) \leq n^{V_\cls}$
\end{block}
\item In other words, the shatter coefficient either grows exactly as $2^n$, or is bounded by a polynomial.
\end{itemize}
\end{frame}

\begin{frame}{Bounds}
\begin{block}{Bound using VC dimension}
If $V_\cls > 2$, then
\[ \Pr{ L(\phi^*_n) - \inf_{\phi \in \cls} (\phi) > \eps } \leq 8 n^{V_c}e^{-n \eps^2/128} \]
Also
\[ \E{L(\phi^*_n)} - \inf_{\phi\in\cls} L(\phi) \leq 16 \sqrt{\frac{V_\cls \log(n) + 4}{2n}} \]
\end{block}
\end{frame}

\begin{frame}{Bounds}
\begin{itemize}
\item In particular, this means that with probability $1-\delta$
\[ L(\phi^*_n) - \inf_{\phi\in\cls} L(\phi) \leq \mathcal{O}\pa{\sqrt{\frac{V_\cls}{n} \log n  + \frac1n \log\frac1\delta}} \]
\item It turns out something a little stronger is true:
\[ L(\phi^*_n) - \inf_{\phi\in\cls} L(\phi) \leq \mathcal{O}\pa{\sqrt{\frac{V_\cls}{n} \log\frac{n}{V_\cls}+ \frac1n \log\frac1\delta}} \]
\end{itemize}
\end{frame}


\begin{frame}{Classifier selection}
\begin{itemize}
\item For a given algorithm of selecting a classifier from data, we also would like to know how much data we need to ensure a certain level of accuracy with certain confidence
\item We say that $N(\eps,\delta)$ is the \emph{sample complexity} of an algorithm if it is the smallest integer such that if $n \geq N(\eps,\delta)$, then if $g_n$ is the selected classifier,
\[ \sup_{(X,Y)} \Pr{ L(g_n) - \inf_{\phi \in \cls} L(\phi) > \eps} \leq \delta \]
whenever $n \geq N(\eps,\delta)$
\begin{block}{Sample complexity of Empirical Risk Minimization}
\[ N(\eps, \delta) \leq \max\pa{ \frac{512 V_\cls}{\eps^2} \log\frac{256V_\cls}{\eps^2}, \frac{256}{\eps^2}\log\frac8\delta}\]
\end{block}
\end{itemize}
\end{frame}

\begin{frame}{Proof of sample complexity - Sketch}
The tricky part is seeing that $n^{V_\cls} \leq e^{n\eps^2/256}$ for $n = \frac{512V_\cls}{\eps^2}\log\frac{256V_\cls}{\eps^2}$, and thus true for all subsequent $n$:
\begin{align*}
&& \frac{512V_\cls}{\eps^2} \log\frac{256V_\cls}{\eps^2}& \leq \pa{\frac{256V_\cls}{\eps^2}}^2 \\
\Rightarrow && \log\pa{\frac{512V_\cls}{\eps^2} \log\frac{256V_\cls}{\eps^2}}& \leq 2 \log \pa{\frac{256V_\cls}{\eps^2}} \\
\Rightarrow && \log\pa{\frac{512V_\cls}{\eps^2} \log\frac{256V_\cls}{\eps^2}}^{V_\cls}& \leq 2V_\cls \log \pa{\frac{256V_\cls}{\eps^2}} \\
\Rightarrow && \log\pa{\frac{512V_\cls}{\eps^2} \log\frac{256V_\cls}{\eps^2}}^{V_\cls}& \leq \frac{\eps^2}{256} \pa{ \frac{512 V_\cls}{\eps^2} \log \pa{\frac{256V_\cls}{\eps^2}}}\end{align*}
\end{frame}

\section{Generative vs. Discriminative}
\subsection{Discriminative}
\begin{frame}{Discriminative classifiers}
\begin{itemize}
\item So far we have been looking at classifiers that are trying to approximate the Bayes classifier $\Pr{Y|X}$ by minimizing
\[ \hat{L}_n(\phi) \defeq \frac1n \sum_{i=1}^n \I_{\phi(X_i) \ne Y_i} \]
\item These are called \emph{discriminative classifiers}
\end{itemize}
\end{frame}

\begin{frame}{Logistic Regression}
\begin{itemize}
\item For example, let $\cls$ be the class of linear classifiers
\item Then one discriminative model is logistic regression
\begin{block}{Logistic Regression}
\[ \Pr{Y=1|X;\beta,\theta} = \frac1{1+exp(\an{\beta,X} + \theta)} \]
We can write a discriminant for this classifier:
\[ D(X) = \an{\beta,X} + \theta = \sum_{i=1}^d \beta_i X_i + \theta \]
where $1$ is chosen if $D(X) > 0$, and 0 otherwise.
\end{block}
\end{itemize}
\end{frame}

\begin{frame}{Convergence for logistic regression}
\begin{itemize}
\item Let $\phi_n$ minimize $\hat{L}_n(\phi)$
\item We have all this machinery for bounding convergence of linear models, so we can say that with high probability,
\[ L(\phi_n) - \inf_{\phi \in \cls}L(\phi) \leq \mathcal O \pa{ \sqrt{\frac{V_\cls}{n} \log\frac{n}{V_\cls}}} \]
\begin{block}{VC dimension of linear classifiers}
If $\cls$ is the class of linear classifiers in $\R^d$, then $V_\cls = d+1$.
\end{block}
\item Thus the rate of convergence is approximately $\sqrt{\frac dn \log \frac nd}$
\end{itemize}
\end{frame}

\subsection{Generative}
\begin{frame}{Generative classifiers}
\begin{itemize}
\item We could also select a classifier that optimizes some other function, for example one that tries to approximate the joint distribution $\Pr{X,Y} = \Pr{X|Y} \cdot \Pr{Y}$
\item Thinking back to our first idea of how the data was generated, this is precisely trying to model that, and so these models are called \emph{generative classifiers}, and compute $\Pr{Y|X}$ from $\Pr{X,Y}$, typically in a Bayesian way
\end{itemize}
\end{frame}

\begin{frame}{Naive Bayes}
\begin{itemize}
\item If we make a strong (naive) independence assumption amongst the features, we can then say that
\begin{block}{Naive Bayes}
\[ \Pr{Y=1|X} \propto \Pr{Y=1} \prod_{i=1}^d \Pr{X_i | Y=1} \]
Equivalently, naive Bayes decides to label $1$ if the following is greater than 1:
\[ \frac{\Pr{Y=1|X}}{\Pr{Y=0|X}} = \frac{\Pr{Y=1}}{\Pr{Y=0}} \prod_{i=1}^d \frac{\Pr{X_i|Y=1}}{\Pr{X_i|Y=0}} \]
\end{block}
\end{itemize}
\end{frame}

\begin{frame}{A linear classifier?}
\begin{itemize}
\item Notice that also equivalently, the discriminant for naive Bayes is:
\begin{block}{Discriminant for naive Bayes}
\[ D(x) = \log \frac{\Pr{Y=1|X}}{\Pr{Y=0|X}} = \underbrace{\log\frac{\Pr{Y=1}}{\Pr{Y=0}}}_{\theta?} +  \underbrace{\sum_{i=1}^d \log \frac{\Pr{X_i|Y=1}}{\Pr{X_i|Y=0}}}_{\an{\beta,X}?} \]
\end{block}
\item This looks a lot like the definition of a linear classifier, and indeed naive Bayes and logistic regression are a \emph{generative-discriminative pair}, in that abstractly they differ only in what they are trained to do
\end{itemize}
\end{frame}

\begin{frame}{Error for Naive Bayes}
\begin{itemize}
\item Given $n$ datapoints, let $\phi_n$ be the logistic regression, and $\psi_n$ the naive Bayes trained on them
\item Simply due to the difference between generative and discriminative classifiers,
\[ \hat{L}_n(\phi_n) \leq \hat{L}_n(\psi_n) \]
\item Additionally, letting $\phi^*$ and $\psi^*$ be the appropriate classifiers trained on the entire population, then
\[ L(\phi^*) \leq L(\psi^*) \]
\item However, we might want to know the convergence rate for naive Bayes as well
\end{itemize}
\end{frame}

\begin{frame}{Convergence for Naive Bayes}
\begin{block}{Convergence for Naive Bayes (Ng and Jordan, 2002)}
Assume  $\rho_0 \leq \Pr{Y=1} \leq 1 - \rho_0$, and $Var(X_i) \geq \rho_0$ for $i=1,\ldots,d$. Then, with high probability,
\[ L(\psi_n) \leq L(\psi^*) + G\pa{\mathcal O\pa{\sqrt{\frac1n \log d}}} \]
where
\begin{align*}
 G(\tau)   = &  \Pr{\br{(X,1) : D_{\psi^*}(X) \in [0,\tau d]}} \\
 	&   + \Pr{ \br{(X,0) : D_{\psi^*}(X) \in [-\tau d,0]}} \end{align*}
in other words, the probability of $\psi^*$ coming within $\tau d$ of making an error
\end{block}
\end{frame}

\begin{frame}{Convergence for Naive Bayes}
\begin{itemize}
\item Thus if $G(\tau) \leq \mathcal{O}(\tau)$, we have a convergence rate of $\sqrt{\frac1n}$, \emph{better than logistic regression}
\item This bound on $G(\tau)$ is plausibly true in the real world
\end{itemize}
\end{frame}

\subsection{Comparison of Error}
\begin{frame}{Tradeoff}
\begin{itemize}
\item Asymptotically, logistic regression performs better than naive Bayes as a classifier
\item However, for small $n$, naive Bayes might actually do better, due to possibly faster convergence
\end{itemize}
\end{frame}

\section{Conclusion}
\begin{frame}{Conclusions}
\begin{itemize}
\item We would like to be able to bound the error of the classifier we choose from a class $\cls$, and we can do so...
\begin{itemize}
\item ...when $\abs{\cls}$ is finite
\item ...when $\cls$ has finite fingering dimension
\item ...when $\cls$ has finite VC dimension
\end{itemize}
\end{itemize}
\end{frame}

\begin{frame}{Counter intuition}
\begin{itemize}
\item This allows us to compare the convergence rate of a discriminative classifier to that of a generative one, yielding a somewhat surprising result:
\item While generative classifiers are not explicitly trained to minimize errors, for smallish $n$ they can out-perform discriminative classifiers
\end{itemize}
\end{frame}


%https://projecteuclid.org/download/pdf_1/euclid.aop/1176988847
%https://ai.stanford.edu/~ang/papers/nips01-discriminativegenerative.pdf

\end{document}
