% !TEX TS-program = pdflatex
% !TEX encoding = UTF-8 Unicode

% This is a simple template for a LaTeX document using the "article" class.
% See "book", "report", "letter" for other types of document.

\documentclass[11pt]{article} % use larger type; default would be 10pt

\usepackage[utf8]{inputenc} % set input encoding (not needed with XeLaTeX)

%%% Examples of Article customizations
% These packages are optional, depending whether you want the features they provide.
% See the LaTeX Companion or other references for full information.

%%% PAGE DIMENSIONS
\usepackage{geometry} % to change the page dimensions
\geometry{a4paper} % or letterpaper (US) or a5paper or....
% \geometry{margin=2in} % for example, change the margins to 2 inches all round
% \geometry{landscape} % set up the page for landscape
%   read geometry.pdf for detailed page layout information

\usepackage{graphicx} % support the \includegraphics command and options

% \usepackage[parfill]{parskip} % Activate to begin paragraphs with an empty line rather than an indent

%%% PACKAGES
\usepackage{booktabs} % for much better looking tables
\usepackage{array} % for better arrays (eg matrices) in maths
\usepackage{paralist} % very flexible & customisable lists (eg. enumerate/itemize, etc.)
\usepackage{verbatim} % adds environment for commenting out blocks of text & for better verbatim
\usepackage{subfig} % make it possible to include more than one captioned figure/table in a single float
% These packages are all incorporated in the memoir class to one degree or another...

%%% HEADERS & FOOTERS
\usepackage{fancyhdr} % This should be set AFTER setting up the page geometry
\pagestyle{fancy} % options: empty , plain , fancy
\renewcommand{\headrulewidth}{0pt} % customise the layout...
\lhead{}\chead{}\rhead{}
\lfoot{}\cfoot{\thepage}\rfoot{}

%%% SECTION TITLE APPEARANCE
\usepackage{sectsty}
\allsectionsfont{\sffamily\mdseries\upshape} % (See the fntguide.pdf for font help)
% (This matches ConTeXt defaults)

%%% ToC (table of contents) APPEARANCE
\usepackage[nottoc,notlof,notlot]{tocbibind} % Put the bibliography in the ToC
\usepackage[titles,subfigure]{tocloft} % Alter the style of the Table of Contents
\renewcommand{\cftsecfont}{\rmfamily\mdseries\upshape}
\renewcommand{\cftsecpagefont}{\rmfamily\mdseries\upshape} % No bold!

%%% END Article customizations



\usepackage[table]{xcolor}
\usepackage{geometry}
\usepackage{pdflscape}

%%% The "real" document content comes below...

\title{Translating a Classifier}
\author{Patrick Martin}
%\date{} % Activate to display a given date or no date (if empty),
         % otherwise the current date is printed 

\begin{document}
\maketitle

\section{Overview}



\subsection{The Data}

In order to work on translating a classifier, I need similar data in different languages. The first thing I'm going to use is reddit, starting with March 2017 (and adding others if I need more data), using langdetect (https://pypi.python.org/pypi/langdetect) on comments that have at least 15 unique words. I'm going to try to pull 10k each of Spanish, French, and Italian. Unfortunately this is either slow or sparse, and is taking a while. Hopefully it will finish by tomorrow or something.

Alternatively/additionally, if I can find different-language wikipediae, I can use page category as the classes.

\subsubsection{LID data}

We can make sure the LID worked reasonably well by checking the subreddits represented


\rowcolors{2}{gray!25}{white}
\begin{tabular}{l|c}
\rowcolor{gray!50} Subreddit & Count  \\
argentina & 4,632 \\
mexico & 1,840 \\
podemos\footnote{Spanish political party} & 1,622 \\
chile & 592 \\
vzla & 280 \\
Argaming & 87 \\
Spanish & 80 \\
uruguay & 75 \\
PuertoRico & 50 \\
Colombia & 44 \\

\end{tabular}
\footnotetext[1]{Spanish political party}
\begin{tabular}{l|c}
\rowcolor{gray!50} Subreddit & Count  \\
france & 7,772 \\
Quebec & 1,052 \\
montreal & 197 \\
ParisComments & 116 \\
French & 44 \\
Lyon  & 35 \\
FiascoQc & 34 \\
SquaredCircle\_FR & 32 \\
effondrement & 27 \\
melenchon & 23 
\end{tabular}
\begin{tabular}{l|c}
\rowcolor{gray!50} Subreddit & Count  \\
italy & 7,862 \\
oknotizie & 525 \\
ItalyInformatica & 351 \\
italy\_SS & 327 \\
italygames & 86 \\
perlediritaly & 69 \\
lisolachece & 62 \\
Romania & 61 \\
ItaliaPersonalFinance & 54 \\
ItalyMotori & 40
\end{tabular}

\subsubsection{Subreddit corpora}

Using the data from the LID stuff, we can also just create the corpora by using all the posts from some subreddits. I propose\footnote{We'll fix this later} \\

\begin{tabular}{c}
\rowcolor{gray!50} Spanish  \\
argentina \\
mexico \\
chile \\
vzla \\
uruguay \\
Colombia \\
\end{tabular}
\begin{tabular}{c}
\rowcolor{gray!50} French  \\
france \\
Quebec \\
montreal \\
Lyon  \\
\end{tabular}
\begin{tabular}{c}
\rowcolor{gray!50} Italian  \\
italy \\
\end{tabular}

\section{Classifier}

In order to make this work, we need to have some aspect that we're trying to classify against. One thing that has been done in the past is classification based on subreddit; based on the subreddits represented this could be possible: national subreddit (argentina, france, italy), gaming (Argaming, jeuxvideo, italygames), however maybe not much else. For reference, these are the fields I have to work with for each comment: \\

\begin{tabular}{c c c c c} 
author & author\_flair\_css\_class & author\_flair\_text & body & controversiality \\
created\_utc & distinguished & edited & gilded & id \\
 link\_id & parent\_id & retrieved\_on & score & stickied \\
 subreddit & subreddit\_id \end{tabular}

Here are what I think:
\begin{itemize}
\item Controversiality \\
\emph{Pro:} Should be fairly language independent \\
\emph{Con:} Pretty sparse, only about 3\% of the documents are ``controversial''. Controversial comments might tend to be in English, as well (comments from outsiders)

\item Score: (binned) \\
\emph{Pro:} Should be fairly language independent \\
\emph{Con:} Borderlines between bins might be difficult, also more extreme scores are very sparse (< 0.2\% have scores of at least 100)

\includegraphics[scale=0.5]{spanish_score.png}

\item Gilded: (binary) \\
\emph{Pro:} Should be fairly language independent \\
\emph{Con:} Extremely sparse ( < 0.03\% have been gilded)

\item Subreddit: \\
\emph{Pro:} Definitely influenced by context \\
\emph{Con:} Might be biased either towards games that are available in the given language, or the text will involve a lot of English.

\item Flair (submissions): \\
\emph{Pro:} Human-decided topic labels
\emph{Con:} Submissions are not as common as comments, and the flairs across different subreddits might not align perfectly. Additionally, the title might not represent the content of the link perfectly.
\end{itemize}

The last idea just sort of came to me while looking at the subreddits. Using title text of submissions might be a good side project if this turns promising. To be specific, this is the number of comments in the smallest class for each of \emph{gilded}, \emph{controversial}, and \emph{score}. (For \emph{gilded} and \emph{controversial}, this is the number of positive examples). \\

\begin{tabular}{|c|c|c|c|}
& Gilded&Contro &Score \\
Spanish-sub & 3& 347& 123\\
Spanish & 0& 52& 16\\
French-sub & 1& 423& 70\\
French & 0 & 64& 10\\
Italian-sub & 1& 189& 24\\
Italian & 1& 59& 15\\
\end{tabular}

First, it's pretty clear that gilded isn't going to work well at all. Second, the bins for score might need to be refined.

\section{Initial Classifiers}

I divided my data into 80\% train and 20\% test and ran a couple classifiers on all of the data. Since \emph{score} is multi-class, the default F1 score doesn't make sense, all of these classifiers will be evaluated on the harmonic mean of the fscores for each class\footnote{Is this a real thing?}. I'm not sure if this is actually something that's really used, but it will work for our comparison purposes.

\begin{tabular}{|c|c|c|c|}
\rowcolor{gray!50}\multicolumn{4}{|c|}{naive bayes} \\
& Gilded&Contro&Score \\
Spanish-sub & 0.0100& 0.0001& 0.0002\\
Spanish & 1.0000& 0.0007& 0.0013\\
French-sub & 0.0392& 0.0185& 0.0002\\
French & err& 0.0006& 0.0004\\
Italian-sub & 0.0392& 0.0002& 0.0004\\
Italian & 0.0392& 0.0006& 0.0011\\
\end{tabular}
\quad \rowcolors{2}{gray!25}{white}
\begin{tabular}{|c|c|c|c|}
\rowcolor{gray!50}\multicolumn{4}{|c|}{logistic regression} \\
& Gilded&Contro&Score \\
Spanish-sub & 0.0100& 0.1551& 0.1145\\
Spanish & err& 0.0004& 0.1513\\
French-sub & 0.0198& 0.1889& 0.0644\\
French & err& 0.1523& 0.0079\\
Italian-sub & 0.0392& 0.1902& 0.0038\\
Italian & err& 0.1198& 0.0057\\
\end{tabular}

\end{document}
